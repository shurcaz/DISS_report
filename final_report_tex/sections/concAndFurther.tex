\section{Conclusion}
This paper details the development of a hybrid reconfiguration planning system through the completion of the following sub-objectives:
\begin{itemize}[]
	\item Development of a task planner to create high-level semantic solutions to state reconfiguration.
	\item Development of a motion planner to impose robot capabilities, geometry, and physical restraints on the semantic solution, and implementation of feedback strategies to discard infeasible solutions and continue the search.
	\item Development of state and state reconfiguration plan visualisation functions to create videos of reconfiguration simulations.
	\item Conducting testing of the system through various inputs to analyse performance.
We demonstrate that the system can produce efficient solutions and potentially can be integrated with the robot arm in the lab to complete sub-objective five. Though the implementation of enhanced feedback strategies is needed to improve solution generation time. The base high-level plan for the system has great potential for further development.
\end{itemize}
We demonstrate that the system can produce efficient solutions and potentially can be integrated with the robot arm in the lab to complete sub-objective 5. Though the implementation of enhanced feedback strategies is needed to improve generation time. The base high-level plan for the system has great potential for further development.


\section{Further Work}
This base system provides many opportunities for future work to enhance its capabilities. From minor changes such as support for the movement of multiple modules at once, multiple mobile manipulators or introducing module orientation and connection direction; To major changes like introducing a failure memory to predict physical layer failures, a purpose-built implementation utilising parallel programming, or the modification of the program to work in real-time so it can compensate for a non-stationary environment. To further identify areas of improvement, the next suggested development of the project would be the creation of a function to generate random goal states from a starting state, to be input into the system. This would allow the automation and conduction of mass testing to develop a data set for analysis.
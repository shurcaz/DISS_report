\section{Overview of Modular Spacecraft}
Modular spacecraft represent a design concept where the overall space system consists of interchangeable modules, each fulfilling specific functions such as propulsion, communication, power generation, or sensing. These standardised modules enable easy assembly to form a unified system, allowing for module movement or replacement to optimize craft efficiency and extend system lifespan. Adopting a modular design approach offers several advantages over traditional methods, including enhanced flexibility, adaptability, and simplified maintenance.
\\\\
Modules feature standardised interfaces that govern physical and electronic interactions, facilitating seamless integration of modules with different purposes or manufacturers into the overall system architecture. While module sizes and shapes may vary across designs, standardisation principles ensure compatibility for integration. The scalability of modular space system architectures depends on the types and quantities of modules used, providing versatility and cost-effectiveness as the system can be tailored to suit specific mission requirements without necessitating a complete redesign.

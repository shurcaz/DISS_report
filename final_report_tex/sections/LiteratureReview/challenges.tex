\subsection{Challenges and Limitations of Automated Reconfiguration in Space}
The limited deployment of complex automated systems, like automated reconfiguration systems, in space is not due to a lack of interest, but rather stems from the formidable technical challenges and high-risk nature of space missions, which cannot afford failures due to their high cost and critical objectives.
\\\\
Space systems must exhibit high reliability and operate effectively across a wide range of conditions. As system complexity increases, so does the number of potential failure points, making the validation, verification, and deployment of complex systems in the space industry a lengthy and costly process. Challenges that autonomous space systems face include:
\\
\begin{itemize}[]
	\item \textbf{Communication latency:} Delays in communications render real-time human intervention impossible, necessitating autonomous systems capable of operating independently without human oversight. Unlike terrestrial applications like self-driving cars that operate under human supervision, autonomous space systems must meet stringent autonomous reliability requirements.
	\\
	\item \textbf{Safety Requirements:} Systems will often be hosting valuable scientific equipment while operating in harsh, unpredictable environments with various hazards such as extreme temperature fluctuations, radiation, space debris, ice, and microgravity.
	\\
	\item \textbf{Limited Power Sources:} Autonomous systems rely on power sources that may not be constant or reliable. For instance, solar-powered crafts may experience power loss during eclipses or due to unexpected collisions with space debris. Autonomous systems must be capable of recovering from temporary power losses or have reliable backup power sources to prevent mission failure.
	\\
	\item \textbf{Isolation:} Unlike on Earth, space missions lack immediate external assistance or observation. Autonomous systems must possess robust sensing capabilities to self-diagnose issues, detect anomalies, and suspend standard operations when necessary to prevent further damage.
	\\
\end{itemize}
Overcoming these challenges demands cutting-edge technology, which has only recently become available, motivating research projects like this one. As computational power and materials sciences advance, we can expect a significant increase in autonomous systems within the space industry in the coming decades.